\documentclass{article}
\title{
Fiber Optic Seismic Aquisition Using Unsupervised Learning}
\author{Samuel Gende }
\date{April 2024}

\begin{document}

\maketitle

\paragraph{Seismic waves are vibrations that travel through the earth. Using different sensors geologists can measure surface waves to monitor earthquake motion, landslide risk, sinkholes, and permafrost. Usually these sensors are costly and thus can't be deployed very densely. Martin Et al. tested fiber-optic distributed acoustic sensing(DAS) arrays, to measure surface waves with fiber optic cables. This approach is cheaper, and has long been used in the oil and gas industry; with cables that are fastened to the ground. Martin Et al. used unfastened cables which produce noisier measurements but are easier to deploy. After collecting data from the DAS array, they extracted features by performing continuous wavelet transforms(CWT). Once the features where extracted they could cluster the data and, mute the noisy problem cluster. This improved the virtual-source response estimate convergence rate, which allows geologist to retrieve data that mimics a controlled-source experiment. }

\paragraph{The original papers unsupervised machine learning technique has four main parts that need to be reimplemented, continuous wavelet transforms, clustering, filtering, and virtual-source response extraction. The first step is performing continuous wavelet transforms on the original signal collected by the DAS array to extract features for clustering. Half of the 60 CWT scale factors where computed on the time axis and the other half on the space axis. After extracting the features the authors then clustered the data using k-means with minibatch optimization. This resulted in 4 clusters one of which the authors identified as high frequency noise from automobiles. To filter out this noise on unseen data the authors performed CWTs on the data. Then muted the high frequency scales that seperated the noisy cluster from other clusters, and finally performed inverse CWTs to return to the original signal. The last step in the proccess was to extract the virtual response estimate, using 1-bit cross correlation. The convergene of the virtual source estimate was used to indicate wether the filtering was useful. If the virtual source estimate converges then the filtering resulted in a more repeatable signal. After these steps have been implemented and tested there is room to test different techniques of clustering, CWTs wavelets, and virtual source retrival. }

\end{document}